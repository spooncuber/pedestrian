%%%%%%%%%%%%%%%%%%%%%%%%%%%%%%%%%%%%%%%%%%%%%%%%%%%%%%%%%%%%%%%%%%%%%%%%%%%%%%%%
%2345678901234567890123456789012345678901234567890123456789012345678901234567890
%        1         2         3         4         5         6         7         8

\documentclass[letterpaper, 10 pt, conference]{ieeeconf}  % Comment this line out if you need a4paper

%\documentclass[a4paper, 10pt, conference]{ieeeconf}      % Use this line for a4 paper

\IEEEoverridecommandlockouts                              % This command is only needed if 
                                                          % you want to use the \thanks command

\overrideIEEEmargins                                      % Needed to meet printer requirements.

% See the \addtolength command later in the file to balance the column lengths
% on the last page of the document

\makeatletter
\let\NAT@parse\undefined
\makeatother

% The following packages can be found on http:\\www.ctan.org
%\usepackage{graphics} % for pdf, bitmapped graphics files
%\usepackage{epsfig} % for postscript graphics files
%\usepackage{mathptmx} % assumes new font selection scheme installed
%\usepackage{times} % assumes new font selection scheme installed
%\usepackage{amsmath} % assumes amsmath package installed
%\usepackage{amssymb}  % assumes amsmath package installed
\usepackage{xspace}
\usepackage{amsmath,amsfonts,amssymb}
\usepackage{graphicx}
%\usepackage{psfig}
%\usepackage{color}
\usepackage{tikz}
\usetikzlibrary{arrows,shapes}
\usetikzlibrary{patterns}
\usepackage[numbers,sort&compress]{natbib}
%\usepackage[backend=bibtex, natbib=true, style=numeric,firstinits=true, url=false, doi=false, isbn=false, maxbibnames=5]{biblatex}

\usepackage{mdwlist}
%\usepackage{enumitem}
\usepackage[export]{adjustbox}

\usepackage{ifthen,version}
\newboolean{include-notes}
%\usepackage[usenames,dvipsnames]{color}
% Comment out the following line to exclude the notes.
\setboolean{include-notes}{false}
\newcommand{\rknote}[1]{\ifthenelse{\boolean{include-notes}}%
 {\textcolor{Dandelion}{\textbf{RK: #1}}}{}}
\newcommand{\wtnote}[1]{\ifthenelse{\boolean{include-notes}}%
 {\textcolor{ForestGreen}{\textbf{WT: #1}}}{}}
\newcommand{\aanote}[1]{\ifthenelse{\boolean{include-notes}}%
 {\textcolor{SkyBlue}{\textbf{AA: #1}}}{}}


%% Modify caption size to 10pt
\usepackage{caption}
\DeclareCaptionFormat{capfont}{\fontsize{10}{6}\selectfont#1#2#3}
\captionsetup{format=capfont}
\usepackage{subcaption}
\captionsetup[subfigure]{format=capfont}

\usepackage[normalem]{ulem}
\usepackage{longtable}
\usepackage{multirow}
\usepackage{booktabs}

% Insert hyperlinks in the document
\usepackage{hyperref}
\hypersetup{bookmarksopen,bookmarksnumbered,
pdfpagemode=UseOutlines,
colorlinks=true,
linkcolor=blue,
anchorcolor=blue,
citecolor=blue,
filecolor=blue,
menucolor=blue,
urlcolor=blue
}

\newcommand{\ind}[1]{\ 1\hspace{-2.3mm}{1}{\left\{#1\right\}}}

\newcommand{\figref}[1]{Figure~\ref{#1}}
\newcommand{\secref}[1]{Section~\ref{#1}}
\newcommand{\thmref}[1]{Theorem~\ref{#1}}
\newcommand{\lemref}[1]{Lemma~\ref{#1}}
\newcommand{\defref}[1]{Definition~\ref{#1}}
\newcommand{\corref}[1]{Corollary~\ref{#1}}
\newcommand{\exref}[1]{Example~\ref{#1}}


%\newtheorem{theorem}{Theorem}
%\newtheorem{lemma}[theorem]{Lemma}
%\newtheorem{ulemma}[theorem]{Utility Lemma}
%\newtheorem*{conc}{Conclusion}
%\newtheorem{definition}{Definition}
%\newtheorem{claim}[theorem]{Claim}
%\newtheorem{remark}{Remark}[section]
%\newtheorem{corollary}[theorem]{Corollary}
%\newtheorem{assumption}[theorem]{Assumption}
%\newtheorem{proposition}[theorem]{Proposition}
%\newtheorem{condition}[theorem]{Condition}
%\newtheorem{example}{Example}[section]



%\newtheorem{corollary}[theorem]{Corollary}

%\newenvironment{proof}[1][Proof]{\begin{trivlist}
%\item[\hskip \labelsep {\bfseries #1}]}{\end{trivlist}}
%\newenvironment{definition}[1][Definition]{\begin{trivlist}
%\item[\hskip \labelsep {\bfseries #1}]}{\end{trivlist}}

%\newcommand{\qed}{\nobreak \ifvmode \relax \else
  %    \ifdim\lastskip<1.5em \hskip-\lastskip
    %  \hskip1.5em plus0em minus0.5em \fi \nobreak
      %\vrule height0.5em width0.6em depth0.25em\fi}


\newcommand{\etal}{{\it et al}}

\newcommand{\half}{\frac{1}{2}}
\newcommand{\thalf}{\frac{1}{2}}
\newcommand{\reals}{\ensuremath{\mathbb{R}}}
\newcommand{\Z}{\ensuremath{\mathbb{Z}}}
\newcommand{\R}{\ensuremath{\mathbb{R}}}
\newcommand{\one}[1]{\left[\!\left[ #1 \right]\!\right]}
\newcommand{\argmin}[1]{\underset{#1}{\mathrm{argmin}} \:}
\newcommand{\argmax}[1]{\underset{#1}{\mathrm{argmax}} \:}
\newcommand{\E}[1]{{\mathbb{E}\left[{#1}\right]}}
\newcommand{\Xcal}{\mathcal{X}}
\newcommand{\Ycal}{\mathcal{Y}}
\newcommand{\Zcal}{\mathcal{Z}}

\newcommand{\hE}[1]{{\mathbb{\hat{E}}\left[{#1}\right]}}
\newcommand{\Ep}[2]{{\mathbb{E}_{#1}\left[{#2}\right]}}
%\DeclareMathOperator{\sign}{sign}
%\DeclareMathOperator{\conv}{conv}
\newcommand{\aO}{\tilde{O}}
\newcommand{\bigO}[1]{{{\cal O}\left({#1}\right)}}
\newcommand{\abigO}[1]{{\aO\left({#1}\right)}}
\newcommand{\hinge}[1]{[#1]_+}
\newcommand{\ip}[2]{{\left\langle{#1},{#2}\right\rangle}}
\newcommand{\inner}[1]{{\left\langle{#1}\right\rangle}}
\newcommand{\abs}[1]{\left\lvert{#1}\right\rvert}
\newcommand{\norm}[1]{\left\lVert{#1}\right\rVert}
\newcommand{\PR}[1]{\Pr\left({#1}\right)}
\newcommand{\ewp}{\ast}



\newcommand{\shai}[1]{\noindent{\textcolor{red}{\{\em #1\} {\bf --Shai}}}}
\newcommand{\nati}[1]{\noindent{\textcolor{red}{\{\em #1\} {\bf --Nati}}}}
%\newcommand{\comment}[1]{}


% Eric's font commands.
\makeatletter
\DeclareFontFamily{OMX}{MnSymbolE}{}
\DeclareSymbolFont{MnLargeSymbols}{OMX}{MnSymbolE}{m}{n}
\SetSymbolFont{MnLargeSymbols}{bold}{OMX}{MnSymbolE}{b}{n}
\DeclareFontShape{OMX}{MnSymbolE}{m}{n}{
    <-6>  MnSymbolE5
   <6-7>  MnSymbolE6
   <7-8>  MnSymbolE7
   <8-9>  MnSymbolE8
   <9-10> MnSymbolE9
  <10-12> MnSymbolE10
  <12->   MnSymbolE12
}{}
\DeclareFontShape{OMX}{MnSymbolE}{b}{n}{
    <-6>  MnSymbolE-Bold5
   <6-7>  MnSymbolE-Bold6
   <7-8>  MnSymbolE-Bold7
   <8-9>  MnSymbolE-Bold8
   <9-10> MnSymbolE-Bold9
  <10-12> MnSymbolE-Bold10
  <12->   MnSymbolE-Bold12
}{}

\let\llangle\@undefined
\let\rrangle\@undefined
\DeclareMathDelimiter{\llangle}{\mathopen}%
                     {MnLargeSymbols}{'164}{MnLargeSymbols}{'164}
\DeclareMathDelimiter{\rrangle}{\mathclose}%
                     {MnLargeSymbols}{'171}{MnLargeSymbols}{'171}
\makeatother

\newcommand{\multiminimax}[1]{\ensuremath{\left\llangle #1\right\rrangle}}
\newcommand{\s}{\mathbf{s}}
\newcommand{\Ldim}{ldim}

\definecolor{Orange}{rgb}{1,0.5,0}
\definecolor{Green}{rgb}{0.5,1.0,0.5}
\definecolor{Blue}{rgb}{0.5,0.5,1.0}
\definecolor{Red}{rgb}{1,0.5,0.5}
\newcommand{\todo}[1]{\textsf{\textbf{\textcolor{Orange}{[[#1]]}}}}
\newcommand{\rmk}[2]{\textcolor{red}{\textbf{#1}:\ [[#2]]}}


%\setlength{\parskip}{5mm plus4mm minus5mm}

%Unmanned Vehichles, adapted from UAVs
\newcommand{\CPS}{AV\xspace}
%\newcommand{\CPS}{LAM\xspace}
\newcommand{\CPSs}{{\CPS}s\xspace}

% Editing markup commands
\newcommand{\repl}[2]{{\color{red}\sout{#1}} {\color{blue} #2}}
\newcommand{\rem}[1]{{\color{red}\sout{#1}}}
\newcommand{\ins}[1]{{\color{green}\underline{#1}}}

\linespread{0.97}


\title{\LARGE \bf
Title Goes Here*
}


\author{Zhi Qiao$^{1}$ and Ross A. Knepper$^{2}$% <-this % stops a space
\thanks{*This work was supported by the HIT Robot Group}% <-this % stops a space
\thanks{$^{1}$Author is with Faculty of Electrical Engineering, Mathematics and Computer Science,
        University of Twente, 7500 AE Enschede, The Netherlands
        {\tt\small author@papercept.net}}%
\thanks{$^{2}$Ross A. Knepper is with the Department of Computer Science, Cornell University, Ithaca, NY 14853, USA
        {\tt\small rak@cs.cornell.edu}}%
}


\begin{document}



\maketitle
\thispagestyle{empty}
\pagestyle{empty}


%%%%%%%%%%%%%%%%%%%%%%%%%%%%%%%%%%%%%%%%%%%%%%%%%%%%%%%%%%%%%%%%%%%%%%%%%%%%%%%%
\begin{abstract}

Based on two pedestrian samples, we present a model to explain the reason why sometimes pedestrians change their trajectories rather than walking as a straight line. We label the influencer in the dateset and obtain a probability model with 2-D KDE. With this model, we can calculate the relative influence likelihood of other pedestrians. We also present a method to predict if there is any pedestrian out of screen and his possible position. At last, we apply our model to the testing video and evaluate the model.

\end{abstract}


%%%%%%%%%%%%%%%%%%%%%%%%%%%%%%%%%%%%%%%%%%%%%%%%%%%%%%%%%%%%%%%%%%%%%%%%%%%%%%%%

\section{INTRODUCTION}

Most pedestrians will choose walk as a straight line in a no-crowded space, because we always choose the shortest path to get our destination. But in a crowded space, pedestrians will change their trajectories because of interaction with others. In most cases, there are two explanations for this, pedestrians change their trajectories to avoid someone in front of them or return to their original trajectories after avoiding a pedestrian. Certainly, sometimes pedestrians will make some small actions for other reason, such as keeping safe distance from others. And some pedestrians just appear incompetent while walking.

In our work, we want to figure out why pedestrians yield instead of walking as a straight line. We look into the dataset of ETH Walking Pedestrians(EWAP) and classify the agents by trajectories. Most agents only have one clear inflection point in the whole trajectory as infinite view. Our approach selects all the agents with one inflection point by a trajectory-dividing algorithm, and transforms their trajectories and other agents�� position when the inflection points happen to a same coordinate system. From the position of other agents at the same time of inflection point, we note that there are a lot of agents close to inflection point on the turn side. This phenomenon happens because pedestrians always change their trajectories after passing another pedestrian to avoid confusion. But not every other pedestrian is influencing focus agent, we need to focus the interaction between pedestrians.

We labeled the influencer of focus agents by watching the video directly. These wandering agents, staying agents and incompetent agents aren��t considered by us. After the transformation, we obtain a distribution of influencers�� point in the view of focus agent by 2-D KDE. The two dimensions are bearing and range. By this model, we make a video to show the likelihood of influencer with another dataset. In this video, blue cross is the focus agent, red square means highest likelihood influencer and green squares mean other influencer. The length of square means the relative likelihood weather this agent is influencer. We can see the reason clearly why pedestrian yield while walking by these square.

In the meanwhile, we also present a method to forecast the existence of pedestrian we can��t see. For example, if we observe a turn of focus pedestrian and there is no agent around. We will say there maybe another pedestrian in the front of him. In the video, the square on the edge of screen mean the likelihood weather there is a pedestrian going to walk in the screen.
Definitions and Problem Statement


\section{RELATED WORK}

TBD

\section{METHOD}


When a pedestrian yields rather than walking as straight line, there are usually some other pedestrians around. Assume the focus agent is agent F, and his trajectory is T. On the trajectory T, there are {n} infection points I={I1,I2,...} break trajectory to {n-1} broken lines L={L1,L2,...}. And when the infection point {In} happens, there are other pedestrians M = {M1,M2,...} with the absolute position P = {Pn1,Pn2,...} [figure 2]. First we need to fit the trajectory with broken lines to classify the agents.

\subsection{Basic Approach}

One of the most important method is classifying pedestrian by the number of infection point and the angle of turning. The tracking frames of the dataset we use is 2.5, we make straight line every three point to get rid of subtle angle difference between adjacent points. Then, we calculate the angle difference between every two lines to create a upper triangular matrix S. By this matrix, we can divide the trajectory to broken lines easily.

If we only want to deal with trajectory with one infection point, we need to find the maximal absolute value in S. It means that we find the maximal angle difference between every two lines, so we can use the two lines to fit trajectory as two broken lines. If we want to fit trajectory with more infection points, what we need to do is finding all the local maximal(minimal) values in the matrix. Every row or line of local maximal(minimal) value in the S is an infection point. But sometimes this method will obtain incorrect infection points as subtle noise in the dataset, we use another scenario to filter incorrect infection points.

After obtaining all the infection points, we put the absolute angles(at absolute coordinate system) of these lines into a vector A and set a threshold Q. The scenario chooses first element of vector A as reference point and searches the vector until it finds an value��s difference with reference point��s value bigger than threshold Q. Then the scenario will regard this point as correct infection point and take this point as reference point to continue searching until finish the whole vector[scenario 1]. After searching, all the reference points are infection points {I} of this trajectory.

This approach solves the trajectory��s fitting-line problem successfully if the trajectory behaves like a few of broken lines. But if the trajectory behaves like a arc with small curvature, the approach won��t work well because it��s difficult to fit arc with a few lines[figure 3].

To observe the relative position between focus agent and other agents�� position at the infection points time, we need to transfer all the focus agent with one infection point to a same coordinate system. We transform the first fitting line L1 to negative X-axis to make sure focus agent come from -X to +X and move the first infection point I1 to (0,0)[transform formula]. Other agents�� position {P} need to be transformed as the same matrix and we mirror the agents which turn right to make them turn right.

Now we get a map about focus agents�� trajectories and other pedestrians�� position when the infection point happens[figure 4]. We can see that there are a lot of pedestrians gather around (0,1). People always turn after passing another pedestrian in the direction they��re going to turn, because they don��t want confuse other pedestrian while walking. This reason make other pedestrians gather around in the direction which focus agent will turn and near to the infection point.

But in figure 4, there are still so many pedestrians in the other place to make results cluttered. If we only want see where the real influencers are, what we need to do is labeling them in the video.


\subsection{Training}

First we need to know the judgment about influencers. Pedestrians�� turning usually cause by various reasons, but two of them happen most : a. pedestrian turn when he find another pedestrian who��s in front of him; b. pedestrian turn after he passing another pedestrian who block his way. From this two case, we can conclude two rules about judging influencer: a. If focus agent don��t turn and keep the direction of first broken line L1, focus agent will collide the front pedestrian, we regard the front pedestrian as an influencer; b. At the same time of I2, if another pedestrian block the straight way from infection point I1 to I3, we regard the blocked pedestrian as an influencer.[figure 5]

With the two rules above, we label the video of ETH by watching the behaviors of every pedestrian in the dataset. We have 420 agents in the dataset, but 242 of them are lost or have no interaction with others, so we have 178 valuable samples totally. These samples need to be transformed by the same transformation as basic approach.

We classify all the samples with two broken lines by the angle between them. Then we get two figures about the relative position between influencers and focus agent[figure 6]. We observe that most influencers are at the inner position of focus agent's turning or in the front of focus agent.


\subsection{Model}

If we only see the position of other pedestrians, it will be too messy to obtain useful result. So we use the KDE (kernel density estimate) to build a distribution about the chance of another pedestrian to be influencer at a special position. We take the euclidean distance between sample points and reference points as the metric of KDE method. The probability of a pedestrian to be an influencer at reference point is [formula 2].

With this distribution, we can build a contour to observe the case that where is the most likely part to have influencer when focus agent turns. We can also apply this model to a video to judge which pedestrian is the most likely influencer to a special pedestrian. In the testing video, we label the focus agent with a blue cross and his trajectory with blue lines. All the other pedestrians in the dataset are labeled with cube(some pedestrians in video arn't labeled because they're lost in the dateset) and red cube means the most likely pedestrian[figure 7].


\subsection{Prediction}

Sometimes we see a pedestrian turn and there's no other pedestrians around, we can infer that there maybe other pedestrians out of our view. Our model can be applied to predict the case out of view by this idea. We expand the broken line {Li} to make an intersection point with the boundary of the view, and assume that there's a pedestrian at the point. After calculate the likelihood of how likely the hypothetical pedestrian will be an influencer, if the likelihood is the most big one, we will judge that there's a pedestrian out of view.

%%%%%%%%%%%%%%%%%%%%%%%%%%%%%%%%%%%%%%%%%%%%%%%%%%%%%%%%%%%%%%%%%%%%%%%%%%%%%%%%

\section{EVALUATION}

TBD



%%%%%%%%%%%%%%%%%%%%%%%%%%%%%%%%%%%%%%%%%%%%%%%%%%%%%%%%%%%%%%%%%%%%%%%%%%%%%%%%
\addtolength{\textheight}{-12cm}   % This command serves to balance the column lengths
                                  % on the last page of the document manually. It shortens
                                  % the textheight of the last page by a suitable amount.
                                  % This command does not take effect until the next page
                                  % so it should come on the page before the last. Make
                                  % sure that you do not shorten the textheight too much.
%%%%%%%%%%%%%%%%%%%%%%%%%%%%%%%%%%%%%%%%%%%%%%%%%%%%%%%%%%%%%%%%%%%%%%%%%%%%%%%%

\section*{ACKNOWLEDGMENT}

The preferred spelling of the word �acknowledgment� in America is without an �e� after the �g�. Avoid the stilted expression, �One of us (R. B. G.) thanks . . .�  Instead, try �R. B. G. thanks�. Put sponsor acknowledgments in the unnumbered footnote on the first page.

%%%%%%%%%%%%%%%%%%%%%%%%%%%%%%%%%%%%%%%%%%%%%%%%%%%%%%%%%%%%%%%%%%%%%%%%%%%%%%%%

\bibliographystyle{plainnat}
\bibliography{bib,knepper}

\end{document}
